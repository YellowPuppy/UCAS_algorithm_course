\documentclass[a4paper,11pt]{article}
\usepackage{graphicx}


%opening
\title{CS612 Assignment 3}
\author{Institute of Computing Technology, \\
                       Chinese Academy of Sciences, Beijing, China }

\begin{document}

\maketitle

Notice:\\\\
1. Due Nov. 19, 2009.\\
2. Please send your answer to wangchao1987@ict.ac.cn, shaomingfu@gmail.com, yuanxiongying@ict.ac.cn\\
3. You can choose two problems from Problem 1-4.
\section{Divide and Conquer (10 marks)}

You are interested in analyzing some hard-to-obtain data from two separate databases. Each database contains $n$ numerical values-so there are $2n$ values total-and you may assume that no two values are the same. You'd like to determine the median of this set of $2n$ values, which we will define here to be the $n^{th}$ smallest value.\\

However, the only way you can access these values is through $queries$ to the databases. In a single query, you can specify a value $k$ to one of the two databases, and the chosen database will return the $k^{th}$ smallest value that it contains. Since queries are expensive, you would like to compute the median using as few queries as possible.\\

Give an algorithm that finds the median value using at most $O($log$n)$ queries.

\section{Divide and Conquer (10 marks)}

Recall the problem of finding the number of inversions. As in the course, we are given a sequence of $n$ numbers $a_1,...,a_n$, which we assume are all distinct, and we difine an inversion to be a pair $i<j$ such that $a_i>a_j$.\\

We motivated the problem of counting inversions as a good measure of how different two orderings are. However, one might feel that this measure is too sensitive. Let's call a pair a $significant\ inversion$ if $i<j$ and $a_i>2a_j$. Given an $O(n$log$n)$ algorithm to count the number of significant inversions between two orderings.

\section{Divide and Conquer (10 marks)}

Consider an $n$-mode complete binary tree $T$, where $n=2^d-1$ for some $d$. Each node $v$ of $T$ is labeled with a real number $x_v$. You may assume that the real numbers labeling the nodes are all distinct. A node $v$ of $T$ is a $local\ minimum$ if the label $x_v$ is less than the label $x_w$ for all nodes $w$ that are joined to $v$ by an edge.\\

You are given such a complete binary tree $T$, but the labeling is only specified in the following $implicit$ way: for each node $v$, you can determine the value $x_v$ by $probing$ the node $v$. Show how to find a local minimum of $T$ using only $O($log$n)\ probes$ to the nodes of $T$.

\section{Divide and Conquer (10 marks)}

Suppose now that you're given an $n\times n$ grid graph $G$. (An $n\times n$ grid graph is just the adjacency graph of an $n \times n$ chessboard. To be completely precise, it is a graph whose node set is the set of all ordered pairs of natural numbers $(i,j)$, where $1\leq i\leq n$ and $1\leq j\leq n$; the nodes $(i,j)$ and $(k,l)$ are joined by an edge if and only if $|i-k|+|j-l|=1$.)\\

We use some of the terminology of the previous question. Again, each node $v$ is labeled by a real number $x_v$; you may assume that all these labels are distinct. Show how to find a local minimum of $G$ using only $O(n)$ probes to the nodes of $G$. (Note that $G$ has $n^2$ nodes.)

\end{document}
