\documentclass[a4paper,11pt]{article}
\usepackage{graphicx}


%opening
\title{CS612 Assignment 8}
\author{Institute of Computing Technology, \\
                       Chinese Academy of Sciences, Beijing, China }

\begin{document}

\maketitle

Notice:\\\\
1. Due Dec. 31, 2009.\\
2. Please send your answer to wangchao1987@ict.ac.cn, shaomingfu@gmail.com, yuanxiongying@ict.ac.cn\\
3. You can arbitrarily choose two problems from Problems 1-5.

\section{Approximation Algorithm(10 marks)}

Consider the following algorithm for (unweighted) {\bf Vertex Cover}: In each connected component of the input graph execute a depth first search (DFS). Output the nodes that are not the leaves of the DFS tree. Show that the output is indeed a vertex cover, and that it approximates the minimum vertex cover within a factor of $2$.

\section{Apptoximation Algorithm(10 marks)}

Given a graph $G={V,E}$ with edge costs and set $T\subseteq V$ of terminal vertices, the $Steiner Tree Problem$ is to find a minimum cost tree in $G$ containing every vertex in $T$ (vertices in $V-T$ may or may not be used in $T$).\\\\
				 (a)Give a $2$-approximation algorithm if the edge costs satisfy the triangle inequality.\\\\
				 (b)Give a $2$-approximation algorithm for general edge costs (The graph also need not be complete).

\section{Approximation Algorithm(10 marks)}

Consider the following maximization version of the 3-Dimensional Matching Problem. Given disjoint sets $X$, $Y$, $Z$, and given a set $T\subseteq X\times Y\times Z$ of ordered triples, a subset $M\subseteq T$ is a $3-dimensional\ matching$ if each element of $X\cup Y\cup Z$ is contained in at most one of these triples. The $Maximum\ 3-Dimensional\ Matching\ Problem$ is to find a 3-dimensional matching $M$ of maximum size. (You may assume $|X|=|Y|=|Z|$ if you want.)\\\\
Give a polynomial-time algorithm that finds a 3-dimensional matching of size at least $ \frac{1}{3} $ times the maximum possible size.

\section{Approximation Algorithm(10 marks)}

Consider an optimization version of the Hitting Set Problem defined as follows. We are given a set $A={a_1,\ a_2,...,a_n}$ and a collection $B_1,\ B_2,...,B_m$ or subsets of $A$. Also, each element $a_i\in A$ has a $weight\ w_i\geq 0$. The problem is to find a hitting set $H\subseteq A$ such that the total weight of the elements in $H$, that is, $\Sigma _{a_i\in H} w_i$, is as small as possible. ($H$ is a hitting set if $H\cap B_i$ is not empty for each $i$.)Let $b=max_i|B_i|$ denote the maximum size of any of the sets $B_1,\ B_2,...,B_m$. Give a polynomial-time approximation algorithm for this problem that finds a hitting set whose total weight is at most $b$ times the minimum possible.

\section{Approximation Algorithm(10 marks)}

Recall that in the Knapsack Problem, we have $n$ items, each with a weight $w_i$ and a value $v_i$. We also have a weight bound $W$, and the problem is to select a set of items $S$ of highest possible value subject to the condition that the total weight does not exceed $W$, that is, $\Sigma _{i\in S}w_i\leq W$. Here's one way to look at the approximation algorithm that we designed in this chapter. If we are told there exists a subset $\vartheta $ whose total weight is $\Sigma _{i\in \vartheta }w_i\leq W$ and whose total value is $\Sigma _{i\in \vartheta}v_i=V$ for some $V$, then our approximation algorithm can find a set $A$ with total weight $\Sigma _{i\in A}w_i\leq W$ and total value at least $\Sigma _{i\in A}v_i\geq V/(1+\epsilon )$. Thus the algorithm approximates the best value, while keeping the weights strictly under $W$.\\\\
Now, as is well known, you can always pack a little bit more for a trip just by ``sitting on your suitcase'', in other words, by slightly overflowing the allowed weight limit. This too suggests a way of formalizing the approximation question for the Knapsack Problem, but it's the following, different, formalization.\\\\
Suppose that you are given $n$ items with weights and values, as well as parameters $W$ and $V$; and you are told that there is a subset $\vartheta$ whose total weight is $\Sigma _{i\in \vartheta}w_i \leq W$ and whose total value is $\Sigma _{i\in \vartheta}v_i=V$ for some $V$. For a given fixed $\epsilon >0$, design a polynomial-time algorithm that finds a subset of items $A$ such that $\Sigma _{i\in A}w_i \leq (i+\epsilon )W $ and $\Sigma _{i\in A}v_i \geq V$. In other words, you want $A$ to achieve at least as high a total value as the given bound $V$, but you are allowed to exceed the weight limit $W$ by a factor of $1+\epsilon $.

\end{document}
