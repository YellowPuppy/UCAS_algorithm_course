%\documentclass[a4paper,11pt]{article}
%\usepackage{graphicx}
%\usepackage{mathrsfs}
%\usepackage{amssymb,amsfonts,amsmath}
%\usepackage{multirow}
%\usepackage{epsfig}
%\usepackage{subfigure}
%
%\def\pdfshellescape{1}
%\usepackage{epstopdf}
%
%\usepackage{color}
%\usepackage{float}
\documentclass[11pt,a4paper,oneside]{article}

% packages
\usepackage[top=25.4mm,bottom=25.4mm,left=31.8mm,right=31.8mm]{geometry}
\usepackage{fancyvrb}   % fancy verbatim
\usepackage{enumerate}  % extra styles for enumerate list environment
\usepackage{setspace}   % setting paragraph spacing
\usepackage{array}      % text wrap in table cell, \multicolumn
\usepackage{multirow}   % \multirow
\usepackage{graphicx}
\usepackage{amsmath}    % using split env, cooperating with ntheorem package
\usepackage[amsmath,amsthm,thmmarks]{ntheorem}
\usepackage[urw-garamond]{mathdesign}
\usepackage[T1]{fontenc}
\newtheorem{thm}{Theorem}[subsection]  % SECTION_NUM.THEOREM_NUM
\newtheorem{cor}[thm]{Corollary}
\newtheorem{lem}[thm]{Lemma}           % SECTION_NUM.LEMMA_NUM
\numberwithin{figure}{section}


\begin{document}

\title{Assignment 3}
\author{Institute of Computing Technology, \\
        Chinese Academy of Sciences, Beijing, China }
\date{\today}
\maketitle

\noindent
\textbf{Notice}
\begin{enumerate}
\item Due date
      \begin{itemize}
      \item Nov. 16, 2012 for CS711008Z.
      \item Nov. 28, 2012 for CS6012.
      \end{itemize}
\item Please send your answer in hard copy.
\item You can choose one problem from Problem 1-6.
\item When you're asked to give an algorithm, you should do at least the following things:
      \begin{itemize}
      \item Describe your algorithm in words or pseudo-code;
      \item Prove that your algorithm can give the right answer;
      \item Analyse the complexity of your algorithm.
      \end{itemize}
\item You can present your answer in English or in Chinese.
\end{enumerate}
%
%

\section{Dynamic Programming}
\noindent
The owners of an independently operated gas station are faced with a following situation. 
They have a large underground tank in which they store gas; 
the tank can hold up to $L$ gallons at one time. 
Ordering gas is quite expensive, so they want to order relatively rarely. 
For each order, 
they need to pay a fixed price $P$ for delivery in addition to the cost of the gas ordered. 
However, it costs $c$ to store a gallon of gas for an extra day, 
so ordering too much ahead increases the storage cost.

They are planning to close for a week in the winter, 
and they want their tank to be empty by the time they close. 
Luckily, based on years of experience, 
they have accurate projections for how much gas they will need each day until this points in time. 
Assume that there are $n$ days left until they close, 
and they need $g_i$ gallons of gas for each of the days $i = 1, 2, \ldots n$. 
Assume that the tank is empty at the end of day $0$. 
Give an algorithm to decide on which days they should place orders, 
and how much to order so as to minimize their total cost.
%
%
\section{Dynamic Programming}
\noindent
Professor Stewart is consulting for the president of a corporation that is planning a company party. 
The company has a hierarchical structure; 
that is, the supervisor relation forms a tree rooted at the president. 
The personnel office has ranked each employee with a conviviality rating, 
which is a real number. 
In order to make the party fum for all attendees, 
the president does not want both an employee and his or her immediate supervisor to attend.

Professor Stewart is given the tree that describes the structure of the corporation, 
{\bf using the left-child, right-sibling representation}. 
Each node of the tree holds, in addition to the pointers, 
the name of an employee and that employee's conviviality ranking. 
Describe an algorithm to make up a guest list that 
maximizes the sum of the conviviality ratings of the guests. 
Analyze the running time of your algorithm.
%
%
\section{Dynamic Programming}
\noindent
Suppose you have one machine and a set of $n$ jobs $a_1, a_2, \dots, a_n$ to process on that machine. 
Each job $a_j$ has a processing time $t_j$, a profit $p_j$, and a deadline $d_j$. 
The machine can process only one job at a time, 
and job $a_j$ must run uninterruptedly for $t_j$ consecutive time units. 
If job $a_j$ is completed by its deadline $d_j$, you receive a profit $p_j$, 
but if it is completed after its deadline, you receive a profit of $0$. 
Give an algorithm to find the schedule that obtains the maximum amount of profit, 
assuming that all processing times are integers between $1$ and $n$. 
What is the running time of your algorithm?
%
%
\section{Dynamic Programming}
\noindent
You're running a computing system capable of processing several terabytes of data per day. 
For each of $n$ days, you're presented with $x_i$ terabytes on day $i$. 
Note that any unprocessed data is abandoned at the end of the day.

However, the computing system can only process a fixed number of terabytes in a given day. 
Besides, 
the amount of data you can process goes down every day since the most recent reboot of the system. 
On the first day after a reboot, you can process $s_1$ terabytes, and so on, up to $s_n$; 
we assume $s_1>s_2>s_3>\cdots>s_n>0$. 
To get the system back to peak performance, you can choose to reboot it; 
but on any day you choose to reboot the system, you can't process any data at all.

Now please give an efficient algorithm that takes values for $x_1$,$x_2$,...,$x_n$ 
and $s_1$,$s_2$,...,$s_n$ and returns the total number of terabytes processed by an optimal solution.
%
%
\section{Dynamic Programming}
\noindent
Given a sequence of $n$ real numbers $a_1,...,a_n$, 
determine a subsequence (not necessarily contiguous) of maximum length in which 
the values in the subsequence form a strictly increasing sequence.
%
%
\section{Dynamic Programming}
\noindent
Write a program to implement the \textsc{Needleman-Wunch} algorithm in your favorite language,
run it over \textit{Gene1\_Seq.txt} and \textit{Gene2\_Seq.txt} using different scoring schemes,
and see how they influence the alignment score.
\\ \\
Note: 
You can design your scoring scheme based on the one given in page $71$ of the lecture slides.

\end{document}
