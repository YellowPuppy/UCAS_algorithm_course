\title{ \LARGE{ {\bf Some Words Before the First Class of Course CS091M4041H/CSB0912009Y} }}
\author{
        Dongbo Bu\\
        \\
         Institute of Computing Technology\\
         Chinese Academy of Sciences \\
         \\
         Email: dbu@ict.ac.cn \\
         WWW: http://bioinfo.ict.ac.cn/$\sim$dbu/ 
}
\date{\today}


\documentclass[12pt]{article}
\begin{document}
\maketitle
\newpage

\section{Course information}
\paragraph{Objective}
The objective of the course can be described as follows: 
\begin{itemize} 
\item  to master the ability to extract mathematically clean core of a problem, 
\item then identify an appropriate algorithm design technique based on the problem structure observations, 
\item  and finally prove the correctness and analyse algorithm performance. 
\end{itemize} 
\paragraph{Web site} 
All the course information, including slides, demos, etc, are available via http://bioinfo.ict.ac.cn/$\sim$dbu/AlgorithmCourses/CS091M4041H/CS091M4041H \_2020.html
\paragraph{TA}
We have a total of seven TAs for the course, and they can be reached at 62600817 or tagc@ict.ac.cn. 

We will have a total of 5 ``Question-and-Answer" time in this term. The actual schedule will be sent to you via email. 

\section{Marking policy}
The final score consists  of the following two parts: 
\begin{enumerate}
\item Assignments (24 marks):  We will have a total of 8 assignments and each assignment has 3 marks.  
\item Final exam or research report (76 marks): The final exam has a total of 10 questions (denoted as $Q1-Q10$). 
\begin{itemize} 
\item $Q1-Q8$: Each question has a mark of 8, and they are simply variants of {\bf randomly chosen} questions from the corresponding assignments. 
\item $Q9-Q10$: Each question has a mark of 6, and they never appear in any assignments in any forms. 
\end{itemize} 
\end{enumerate}

{\bf Notice:}
\begin{enumerate}  
\item Algorithm implementation on computer is highly emphasized in our course besides simply writing pseudo-code on paper. 
\item You would better write answers using Latex and finally submit a pdf file. Latex suites are available through: 
\begin{itemize}
\item Mac system: TexShop (http://download.cnet.com/TeXShop/3000-2054\_4-6112.html)
\item Windows system: CTEX (http://www.ctex.org) is a good choice. 
\item Linux system: TexWorks + spell 
\end{itemize} 
\item A template for drawing figures using Latex is available on the course website. 
%\item ``Copy+paste" is {\bf NOT} welcome. 
\end{enumerate} 

%\section{Recommended learning strategy}
%\begin{itemize}
%\item The ``1+3+7" schedule is recommended; that is, you would better spend 1 hour to quickly browse the slides before the class. The objective is to has an idea of the problems rather than algorithms to solve the problems. And after the class, you would better spend 7 hours to review the slides, to do an extension reading,  to implement the algorithms, etc. 
%\item The ``group learning" strategy is widely employed in US universities, i.e., we set up a collection of 5-student groups, and each group member read a paper and present it in group meeting. Thus, all group members will harvest 5 papers in a short time. 
%\end{itemize}


\end{document}

